\documentclass[
    a4paper,
    english,
    DIV=16,
    11pt,
    parskip=half,
]{scrartcl}
\usepackage[
    pdfhighlight=/O,
    colorlinks,
    linkcolor=blue,
    urlcolor=blue,
    citecolor=black,
    filecolor=black,
    breaklinks,
    bookmarksopen,
    bookmarksopenlevel=1,
    linktocpage
] {hyperref}
\usepackage[english]{babel}
\usepackage[dvips]{graphicx}
\usepackage{amsmath}
\usepackage{enumitem}
\usepackage{float}
\usepackage{microtype}
\usepackage{xcolor}
\usepackage{textcomp}
\usepackage{tikz}
\usepackage[
  firstpage=true,
  scale=1,
  angle=0,
  position={current page.north},
  hshift=7.5cm, 
  vshift=-1.0cm,
  opacity=1,
  contents={%
    \includegraphics[width=5cm,keepaspectratio]{../logos/mars_logo.png}%
  }%
]{background}
\usepackage{pstricks}
\usepackage{dirtytalk}
\parindent 0pt
\bibliographystyle{alphadin}
\newcommand\todo[1]{\textcolor{red}{#1}}

\title{WP Artificial Intelligence and Software Agents}
\subtitle{LaserTag Group Competition: Guidelines}
\author{MARS Group}
\date{\today}
\setcounter{secnumdepth}{3}

% ======================================================================

\begin{document}
\maketitle

This outline describes the process leading up to the LaserTag competition and the process on competition day. If you have any questions about these guidelines, please reach out to us in the LaserTag Slack channel.

\section*{Agent Submission and Review}
\begin{itemize}
  \item Your agent types must be submitted by the deadline specified in the LaserTag assignment.
  \item Your agent types must satisfy all constraints listed in the LaserTag assignment and in the LaserTag documentation. Any agent type that does not satisfy all of the constraints cannot participate in the competition.
  \item We will run test simulations with your agent types to make sure that none of them break the game (e.g., by causing exceptions or infinite loops). If your code causes issues during testing, we will let you know. You will have until HH:mm on DD.MM.yyyy to resolve these issues. \todo{Specify}
  \item If you agent types explicitly reference agent team colors (\texttt{Green}, \texttt{Red}, \texttt{Blue}, or \texttt{Yellow}), please submit four versions of your code with different type referencing. We are unable to determine ahead of time which color your team will be assigned at each stage of the competition. Having four versions of your code will allow us to assign your team a color more easily. \todo{Do we still need this?}
\end{itemize}

\section*{Competition Setup and Process}
\begin{itemize}
  \item The competition will take place at LOCATION and begin at HH:mm on DD.MM.yyyy. \todo{Specify}
  \item The competition setup and sequence of games is outlined in the PDF document \texttt{tournament\_table} (see the \texttt{Tournament} directory of the LaserTag repository). The games will take place in the order in which they are numbered. \todo{Create this table and review this point}
  \item A simulation tick is defined as one real-time second. Each game will last xxxx ticks. \todo{Specify number of ticks}
\end{itemize}

\subsection*{Map description}
\begin{itemize}
  \item The maps for the competition are in the \texttt{Maps} directory of the repository. \todo{Should the maps used during the competition be known or unknown ahead of time?}
  \item A game with three teams will be played on a triangle-shaped map.
  \item A game with four teams will be played on a rectangle-shaped map.
  \item The borders of each map are lined with \texttt{Barrier} agents, forcing the agents to remain within the bounds of the environment.
  \item There will be two types of maps:
  \begin{itemize}
    \item \emph{open}: mainly open terrain, similar to the map description found in the LaserTag documentation
    \item \emph{labyrinth}: mainly tight spaces and corridors
  \end{itemize}
\end{itemize}

\section*{Topics for Discussion}
We encourage an open conversation and discussion among the group. Feel free to bring and share your thoughts, opinions, and feedback on the following subjects (and anything else related to artificial intelligence, software agents, and your experience with this course and with LaserTag):
\begin{itemize}
  \item Your agents:
  \begin{itemize}
    \item What are the main ideas that guided your design choices?
    \item What are your agents' behaviors driven by?
    \item What differences did you notice in your thinking and approach during the development of rule-based and learning-based behaviors?
    \item What behaviors are your rule-based agents programmed to make, and why?
    \item How did you specify the hyperparameter values of the RL algorithm used in your learning agents?
    \item Do you notice any differences in your ability to explain the observed behaviors of your rule-based and learning-based agents? If yes, which?
    \item What behaviors could have been programmed but were not, and why?
    \item Which parts of the LaserTag environment do your agents interact with in what ways?
    \item What were some of your challenges (logical and technical) during the development process? How did you overcome these challenges?
  \end{itemize}
  \item LaserTag and the MARS Framework:
  \begin{itemize}
    \item What did or didn't you like about this assignment?
    \item Is there any functionality (or access to already defined methods) that isn't currently available and that you would have liked to have in the game?
    \item Do you have any suggestions for improving the LaserTag game, documentation, or project?
    \item Do you have any suggestions for the MARS Framework in general?
  \end{itemize}
\end{itemize}

\end{document}
